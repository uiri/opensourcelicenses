\documentclass[12pt,letterpaper]{article}
\usepackage{ifpdf}
\usepackage{listings}
\usepackage[parfill]{parskip}
\usepackage[top=1in, bottom=1in, left=1in, right=1in]{geometry}
\usepackage{titlesec}
\usepackage{enumitem}
\usepackage{graphicx}
\usepackage{caption}
\usepackage{wrapfig}

\usepackage{tocbibind}
\usepackage{url}

\renewcommand{\abstractname}{Executive Summary}
\titlespacing*{\section}{0pt}{12pt plus 4pt minus 2pt}{0pt plus 4pt minus 2pt}
\titlespacing*{\subsection}{0pt}{12pt plus 4pt minus 2pt}{0pt plus 4pt minus 2pt}
\begin{document}

\thispagestyle{empty}
% Memo, if needed, goes here...

\begin{titlepage}
\vspace*{1.5in}
\begin{center}
  \huge A Comparison of Open Source Software Licenses
  \vskip1.5in
  \large by \\
  \large William Pearson and Kurt Bruneau \\
\end{center}
\end{titlepage}

\pagenumbering{roman}

\tableofcontents
\listoffigures
\newpage

\begin{abstract}
Stuff about open source software licenses.
\end{abstract}
\newpage
\thispagestyle{empty}
\pagenumbering{arabic}
\setcounter{page}{1}

\section{Introduction}

As we work on personal projects, we have learned the pain of being unable to restore code back to a working state or losing source files altogether. For this reason, we keep each project in a version control repository, often using git. Due to its ease of use and high visibility, GitHub has become a popular place to put up projects which show off one's programming ability and attract bug reports and contributions from others. Unfortunately, not all projects on GitHub have licenses attached to them.

Without an explicit license from the copyright holder, something as innocuous as `git clone' can become copyright infringement. The default model for copyright is all rights reserved. This means that if a copyright holder wants their software to be widely used, improved upon and shared, they should include a license which makes their software free or open source.

\subsection{Free Software}

Free software refers to free in the sense of ``free speech'' rather than free as in the phrase ``free beer''. In order for software to be considered free by the Free Software Foundation, it must grant its users four essential freedoms:

\subsubsection{The four essential freedoms}
\begin{enumerate}
\item[0.] The freedom to run the program as you wish.
\item The freedom to study the program's source code and then change it so the program does what you wish.
\item The freedom to distribute exact copies to others, when you wish.
\item The freedom to distribute copies of your modified versions to others, when you wish.
\end{enumerate}

 In the early days of computing, all software was free by default. Partially, this was an attitude inherited from academia and partially it was because programs would not run on other computers unless they were free and could be ported to those other computers. This began to change with the emergence of the software industry. Bill Gates famously wrote a scathing letter in  1976 accusing personal computer hobbyists of stealing from him. Bill Gates did not make his software free. The Free Software Foundation was a reaction caused by the trend starting in the 1980s of vendors refusing to provide source code to end users.

\subsection{Open Source Software}

In the 1990s, due to persistent confusion around the phrase ``free software'', the Open Source Initiative was started, mainly as a way to sell the pragmatic benefits of free software to businesses without the ethical issues attached to nonfree software by the Free Software Foundation. In order for software to be open source, its distribution terms must meet the open source definition:

\subsubsection{Open Source Definition}
\begin{enumerate}
\item Free Redistribution (must not be restricted)
\item Source Code (must be included)
\item Derived Works (must be allowed)
\item Integrity of the Author's Source Code (may be maintained if patches may be distributed along side it)
\item No Discrimination Against Persons or Groups
\item No Discrimination Against Fields of Endeavor
\item Distribution of License
\item License Must Not Be Specific to a Product
\item License Must Not Restrict Other Software
\item License Must Be Technology-Neutral
\end{enumerate}

\section{Permissive}

\subsection{BSD License}

The term ``BSD License'' can be misleading due to three different versions which have been used historically. For the purposes of this paper, they will be referred to as the ``original'' BSD License, the ``modified'' BSD License and the ``simplified'' BSD License. Some sources may refer to them by the number of clauses; the ``original'' BSD License has four clauses, while the ``modified'' BSD License has three and the ``simplified'' BSD License has only two. Except for the presence or absence of the third and fourth clauses of the ``original'' BSD License, all three versions of the BSD License are essentially the same.

The ``original'' BSD License was written specifically for the Berkeley Standard Distribution (BSD, from which the license derives its name) and listed the University of California as the authors of the software. Its third clause required that all advertising for software covered by the license (BSD and its derivatives) acknowledge that the software being advertised ``includes software developed by the University of California, Berkeley and its contributors.''

This was problematic from the point of view of many in the Free and Open Source Software community, particularly as other projects adopted the ``original'' BSD License with the phrase ``University of California, Berkeley'' replaced by the names of the authors. In 1997, in order to advertise NetBSD, 75 different acknowledgements of this kind would have been required. This absurdity ultimately led to the ``modified'' BSD license.

When FreeBSD removed their advertising clause, they also removed the non-endorsement clause, leading to the ``simplified'' BSD License. The language of the ``simplified'' BSD License has substantially the same legal effect as the language of the MIT License.

\subsection{MIT License}

This license is used by popular open source projects such as Ruby on Rails, Node.js, Lua and jQuery. The license originated at MIT with the X Window System, a GUI for Unix machines which is still present today in modern Unix systems (such as FreeBSD and Mac OS X) and Unix-like systems (such as Ubuntu and other Linux distributions). Originally the license was written specifically for the X Window System and made specific references to the X consortium.

Expat, an XML library for C, genericized the license and it is the license used by Expat which is often taken verbatim (excluding the copyright notice, of course) when one refers to using the MIT license. The license used by X Window System also included a clause similar to the third clause of the modified BSD License. while Expat stripped out this clause. It is for this reason that some, such as the Free Software Foundation, view the term ``MIT License'' as ambiguous. The MIT License is substantially similar to the BSD License, although its terms are not separated out into clauses which makes the license easier and quicker to read.

The MIT License explicitly mentions permission to ``use... and/or sell'' copies of the software. This has been argued to include a patent grant as ordinarily there are no copyright-related restrictions on using and selling a piece of software so long as no copy is made. For instance, if someone receives the software on a CD-ROM, they can freely use and sell the CD-ROM so long as they don't rip the software from it.

\subsection{Apache License}

The original Apache License was essentially the same as the ``original'' BSD License with two clauses added. A clause prohibiting modified versions from using the name ``Apache'' without permission so as to prevent confusion with the original software was added. A clause requiring the advertising acknowledgement be applied to any form of redistribution was also added.

In version 1.1 of the Apache License, the advertising clause was removed. This made it essentially the same as the ``modified'' BSD License with the two Apache clauses added.

Apache License, Version 2.0 brought more substantial changes. It removed explicit references to the Apache Software Foundation to make use of the license by others easier and clarified some language so that others could more easily contribute to projects licensed under the Apache License. This included a patent grant from any contributor whose contribution infinged their own patent.

\subsection{WTFPL and Public Domain}

The Do What the Fuck You Want to Public License is perhaps the most liberal copyright license available, being virtually indistinguishable from dedication to the public domain. In some countries, it is not possible to give up all rights in a copyrighted work and reliably dedicate some work to the public domain. This makes the WTFPL a more suitable alternative for those who wish to allow others to treat their project as if it were in the public domain.

SQLite is an embedded SQL database whose code has been dedicated to the public domain. The company that employs the principle developers provides an option for people to obtain an explicit license to SQLite due to the existence of obtuse legal departments and jurisdictions which do not allow authors to dedicate works to the public domain.

\subsection{Other Permissive Licenses}

Boost, a popular collection of C++ libraries, has its own license which is essentially the same as the MIT License and is substantially similar to the ``simplified'' BSD License.

The ISC License is even less verbose than the MIT License. The language which was removed was deemed obsolete due to the Berne convention. This license has become the preferred license for OpenBSD and other projects. Due to its lack of verbosity, it may be argued that it does not contain the implicit patent grant which the MIT license may contain. The Fair License is substantially similar to the ISC License although far less popular.

The Attribution Assurance License is substantially similar to the BSD License, except that the license is GPG-signed and requires that redistribution include the GPG-signed form of the license. This assures downstream recipients of the software that the terms of the license have not been changed by redistributors.

The Academic Free License includes a copyright and a patent license. It also includes a choice of juridiction clause in favour of the author and includes a provision for recovering legal fees incurred enforcing the terms of the license. It has been deemed redundant due to the creation of the Apache License Version 2.0.

The Education Community License is essentially the same as the Apache License. It has the patent grant changed due to the different needs of the education community when it comes to patent licensing compared with industry needs.

\newpage
\section{Copyleft}

Copyleft licenses are used to keep software both free and open. As opposed to permissive licenses, copyleft license require that all derived works be distributed under the same terms of the original work. This ensures that all derived works maintain the same freedoms. However, as a result of this restriction, many copyleft licenses are often left incompatible with other licenses. \cite{copyleft}

\subsection{GNU GPLv2}

The GNU General Public License License v2 or GPLv2 is the most commonly used free software license today. It is used in various major projects including; Linux, \textbf{Add more projects}.

The GNU GPL, first developed in 1989 by Richard Stallman, was designed to allow for the distribution of the GNU project; ensuring that any derived copies would maintain the same freedoms associated with the project. In 1991 the GPLv1 license was revised to create the GPLv2 license. This revision however was only a rephrase of the original license and did not introduce any new legal implications.

Some of the main features of the GNU GPLv2 include: \cite{gpl2}

\begin{itemize}
\item The ability to copy and redistribute unmodified source code
\item The ability to modify and redistribute modified source code
\item The ability to distribute a compiled version of the software considering that all copies contain a copyright notice and exclusion of warranty.
\item All modified copies are distributed under the GPLv2 license
\item All compiled programs must be accompanied with the corresponding source code or be provided access to the source code.
\end{itemize}

\subsection{GNU GPLv3}

The GNU GPLv3, developed in 2007 was designed to improve on some of the short comings of the original GPLv2 license. Many people view the revision as controversial and criticize it for fracturing the relationship between the open source and free software communities.

One of the major differences between GPLv3 and GPLv2 is that GPLv3 will block ``Tivoization''. Tivoization is a term used to describe appliances that contain GPL'ed software but violate the intended freedoms of the license by making modifications impossible. This is often enforced by shutting down the appliance when modifications are detected.

A common motive for ``Tivoization'' is to prevent users from making modifications when the manufacturer knows the user will want to make changes. Manufactures often claim that this will prevent piracy or that this will somehow provide security, but is often used as a tactic to for a manufacturer to gain a monopoly on some specific market.

\subsection{GNU GPL Variations}

\subsubsection{LGPL}

The GNU Lesser General Public license was designed to allow developers and companies to embed LGPL software into projects not licensed under a GPL license, even if the project contains proprietary software. This difference allows companies to include free software in their projects without being bound by the required terms of a strong copyleft license.

\subsubsection{AGPL}

The GNU Affero General Public License was designed to close a perceived loophole in the original GPL.

\subsection{Others}

\subsubsection{APSL}
\subsubsection{CeCILL}
\subsubsection{MPL}
\subsubsection{IPL}

\section{Limited Copyleft}

\subsection{CDDL}

\section{Conclusion}

\newpage

\begin{thebibliography}{9}
\bibitem{copyleft} GNU Operating System. What is Copyleft? \url{https://www.gnu.org/copyleft/}.

\bibitem{gpl2} GNU Operating System. GNU General Public License, version 2. \url{https://www.gnu.org/licenses/gpl-2.0.html}.

\bibitem{gpl3} GNU Operating System. Why Upgrade to GPLv3. \url{https://www.gnu.org/licenses/rms-why-gplv3.html}.

\end{thebibliography}

% \section*{References}

% \begin{itemize}[label={},itemindent=-15pt]

% \item \openup 0.5em References go here.


% \end{itemize}


\end{document}
