\documentclass[12pt,letterpaper]{article}
\usepackage{ifpdf}
\usepackage{listings}
\usepackage[parfill]{parskip}
\usepackage[top=1in, bottom=1in, left=1in, right=1in]{geometry}
\usepackage{titlesec}
\usepackage{enumitem}
\usepackage{graphicx}
\usepackage{caption}
\usepackage{wrapfig}

\usepackage{tocbibind}
\usepackage{url}

\renewcommand{\abstractname}{Executive Summary}
\titlespacing*{\section}{0pt}{12pt plus 4pt minus 2pt}{0pt plus 4pt minus 2pt}
\titlespacing*{\subsection}{0pt}{12pt plus 4pt minus 2pt}{0pt plus 4pt minus 2pt}
\begin{document}

\thispagestyle{empty}
% Memo, if needed, goes here...

\begin{titlepage}
\vspace*{1.5in}
\begin{center}
  \huge A Comparison of Open Source Software Licenses
  \vskip1.5in
  \large by \\
  \large William Pearson and Kurt Bruneau \\
\end{center}
\end{titlepage}

\pagenumbering{roman}

\tableofcontents
\listoffigures
\newpage

\begin{abstract}
This report compares and contrasts various copyright licenses which meet the criteria for either free or open source software. Two main categories of licenses are explored: permissive licenses which focus on developer freedom and copyleft licenses which focus on end user freedom. This report concludes that despite the plethora of licenses available, software engineers should familiarize themselves mainly with the MIT license, the Apache license and the GNU Genera Public License and refer other licenses to legal professionals.
\end{abstract}
\newpage
\thispagestyle{empty}
\pagenumbering{arabic}
\setcounter{page}{1}

\section{Introduction}

As we work on personal projects, we have learned the pain of being unable to restore code back to a working state or losing source files altogether. For this reason, we keep each project in a version control repository, often using git. Due to its ease of use and consequent popularity, GitHub has become a popular place to put up projects which show off one's programming ability. Its popularity has led to it becoming the host of many mature and fledgling open source projects. Unfortunately, not all projects on GitHub have copyright licenses attached to them.

Without an explicit license from the copyright holder, something as seemingly innocuous as `git clone' can become copyright infringement. The default model for copyright is all rights reserved. This means that if a copyright holder wants their software to be widely used, improved upon and shared, they should include a license which makes their software free or open source.\cite{nolicense}

\subsection{Free Software}

Free software refers to free in the sense of ``free speech'' rather than free as in the phrase ``free beer''. In order for software to be considered free by the Free Software Foundation, it must grant its users four essential freedoms:\cite{fourfreedoms}

\subsubsection*{The four essential freedoms}
\begin{enumerate}
\item[0.] The freedom to run the program as you wish.
\item The freedom to study the program's source code and then change it so the program does what you wish.
\item The freedom to distribute exact copies to others, when you wish.
\item The freedom to distribute copies of your modified versions to others, when you wish.
\end{enumerate}

 In the early days of computing, all software was free by default. Partially, this was an attitude inherited from academia and partially it was because programs would not run on other computers unless they were free and could be ported to those other computers. This began to change with the emergence of the software industry.\cite{fsfhistory} Bill Gates famously wrote a scathing letter in  1976 accusing personal computer hobbyists of stealing from him. Bill Gates did not make his software free.\cite{billyg} The Free Software Foundation was a reaction caused by the trend starting in the 1980s of vendors refusing to provide source code to end users.\cite{fsfhistory}

\subsection{Open Source Software}

In the 1990s, due to persistent confusion around the phrase ``free software'', the Open Source Initiative was started, mainly as a way to sell the pragmatic benefits of free software to businesses without the ethical issues attached to nonfree software by the Free Software Foundation.\cite{opensourcehistory} In order for software to be open source, its distribution terms must meet the open source definition:

\subsubsection*{Open Source Definition\cite{opensourcedefinition}}
\begin{enumerate}
\item Free Redistribution (must not be restricted)
\item Source Code (must be included)
\item Derived Works (must be allowed)
\item Integrity of the Author's Source Code (may be maintained if patches may be distributed along side it)
\item No Discrimination Against Persons or Groups
\item No Discrimination Against Fields of Endeavor
\item Distribution of License
\item License Must Not Be Specific to a Product
\item License Must Not Restrict Other Software
\item License Must Be Technology-Neutral
\end{enumerate}

\section{Permissive Licenses}

Permissive licenses are the licenses which offer the greatest freedom to developers. Developers are generally free to modify and redistribute the osftware subject to few conditions. The most common condition is including the copyright notice and a liability disclaimer with copies of the software.\cite{permissive}

\subsection{BSD License}

The term ``BSD License'' can be misleading due to three different versions which have been used historically.\cite{netbsd} For the purposes of this paper, they will be referred to as the ``original'' BSD License, the ``modified'' BSD License and the ``simplified'' BSD License. Some sources may refer to them by the number of clauses; the ``original'' BSD License has four clauses, while the ``modified'' BSD License has three and the ``simplified'' BSD License has only two. Except for the presence or absence of the third and fourth clauses of the ``original'' BSD License, all three versions of the BSD License are essentially the same.

The ``original'' BSD License was written specifically for the Berkeley Standard Distribution (BSD, from which the license derives its name) and listed the University of California as the authors of the software. Its third clause required that all advertising for software covered by the license (BSD and its derivatives) acknowledge that the software being advertised ``includes software developed by the University of California, Berkeley and its contributors.''\cite{originalbsd}

This was problematic from the point of view of many in the Free and Open Source Software community, particularly as other projects adopted the ``original'' BSD License with the phrase ``University of California, Berkeley'' replaced by the names of the authors. In 1997, in order to advertise NetBSD, 75 different acknowledgements of this kind would have been required. This absurdity ultimately led to the ``modified'' BSD license.\cite{netbsd}

When FreeBSD removed their advertising clause, they also removed the non-endorsement clause, leading to the ``simplified'' BSD License. The language of the ``simplified'' BSD License has substantially the same legal effect as the language of the MIT License.\cite{simplifiedbsd}

\subsection{MIT License}

This license is used by popular open source projects such as Ruby on Rails, Node.js, Lua and jQuery. The license originated at MIT with the X Window System, a GUI for Unix machines which is still present today in modern Unix systems (such as FreeBSD and Mac OS X) and Unix-like systems (such as Ubuntu and other Linux distributions). Originally the license was written specifically for the X Window System and made specific references to the X consortium. \cite{x11}

Expat, an XML library for C, genericized the license and it is the license used by Expat which is often taken verbatim (excluding the copyright notice, of course) when one refers to using the MIT license. \cite{mit} The license used by X Window System also included a clause similar to the third clause of the modified BSD License, while Expat stripped out this clause. It is for this reason that some, such as the Free Software Foundation, view the term ``MIT License'' as ambiguous. \cite{licenselist} The MIT License is substantially similar to the BSD License, although its terms are not separated out into clauses which makes the license easier and quicker to read.

The MIT License explicitly mentions permission to ``use... and/or sell'' copies of the software. This includes an implied patent license as ordinarily there are no copyright-related restrictions on using and selling a piece of software so long as no copy is made.\cite{boost} For instance, if someone receives the software on a CD, they can freely use and sell the CD so long as they don't rip the software from it.\cite{copyright}

\subsection{Apache License}

The original Apache License was essentially the same as the ``original'' BSD License with two clauses added. A clause prohibiting modified versions from using the name ``Apache'' without permission so as to prevent confusion with the original software was added. A clause requiring the advertising acknowledgement be applied to any form of redistribution was also added.\cite{apache1}

In version 1.1 of the Apache License, the advertising clause was removed. This made it essentially the same as the ``modified'' BSD License with the two Apache clauses added.\cite{apache11}

Apache License, Version 2.0 brought more substantial changes. It removed explicit references to the Apache Software Foundation to make use of the license by others easier and clarified some language so that others could more easily contribute to projects licensed under the Apache License. This included a patent grant from any contributor whose contribution infinged their own patent.\cite{apache2}

\subsection{WTFPL and Public Domain}

The Do What the Fuck You Want to Public License is perhaps the most liberal copyright license available, being virtually indistinguishable from dedication to the public domain. In some countries, it is not possible to give up all rights in a copyrighted work and reliably dedicate some work to the public domain. This makes the WTFPL a more suitable alternative for those who wish to allow others to treat their project as if it were in the public domain.\cite{wtfpl}

SQLite is an embedded SQL database whose code has been dedicated to the public domain. The company that employs the principle developers provides an option for people to obtain an explicit license to SQLite due to the existence of obtuse legal departments and jurisdictions which do not allow authors to dedicate works to the public domain.\cite{sqlite}

\subsection{Other Permissive Licenses}

Boost, a popular collection of C++ libraries, has its own license which is essentially the same as the MIT License and is substantially similar to the ``simplified'' BSD License. \cite{boost}

The ISC License is even less verbose than the MIT License. The language which was removed was deemed obsolete due to the Berne convention. This license has become the preferred license for OpenBSD and other projects.\cite{openbsd} Due to its lack of verbosity, it may be argued that it does not contain the implicit patent grant which the MIT license may contain. The Fair License is substantially similar to the ISC License although far less popular.\cite{fair}

The Attribution Assurance License is substantially similar to the BSD License, except that the license is GPG-signed and requires that redistribution include the GPG-signed form of the license. This assures downstream recipients of the software that the terms of the license have not been changed by redistributors.\cite{attrassur}

The Academic Free License includes a copyright and a patent license. It also includes a choice of juridiction clause in favour of the author and includes a provision for recovering legal fees incurred enforcing the terms of the license. It has been deemed redundant due to the creation of the Apache License Version 2.0.\cite{academic}

The Education Community License is essentially the same as the Apache License. It has the patent grant changed due to the different needs of the education community when it comes to patent licensing compared with industry needs.\cite{education}

\section{Copyleft Licenses}

Copyleft licenses are designed to keep software both free and open. As opposed to permissive licenses, copyleft licenses are modelled around the free software philosophy. A copyleft license guarantees that any derived work must be distributed under the same set of freedoms of the original work \cite{copyleft}. While this licensing restriction can greatly improve collaboration in a development ecosystem, it often leads to incompatibilies with other licenses.

\subsection{GNU GPLv2}

The GNU General Public License License v2 is the most common open source license. It is used in a wide range of free and open source projects including the worlds largest open source project, Linux.

Frustrated by the shift in software culture, Stallman, the designer of the GPL, was determined to ensure that users would remain fully in control of their software. The surge in proprietary software challenged Stallman's vision; influencing him to develop the GNU project and the Free Software Foundation.

In the process of developing the GNU project, Stallman realized that his collection of software maintained the freedoms that he set out to ensure, but the fractured licensing of each application prevented his applications from sharing code. As a result, in 1989, he designed the GNU GPL to unify the licenses of the GNU project; allowing him to share the code between each application while maintaining the goals of a copyleft license \cite{gpl2}. Two years later he released the GPLv2 to rephrase the legal language of the original GPL, however this change didn't introduce any new legal implications.

Some of the main features of the GNU GPLv2 include:
\begin{enumerate}
\item The ability to copy and redistribute unmodified source code
\item The ability to modify and redistribute modified source code
\item The ability to distribute a compiled version of the software considering that all copies contain a copyright notice and exclusion of warranty.
\item All modified copies are distributed under the GPLv2 license
\item All compiled programs must be accompanied with the corresponding source code or be provided access to the source code.
\end{enumerate}

\cite{gpl2}

These features effectively allow anyone to use, copy, and distribute software while ensuring that the original author is protected. For instance, the exclusion of warranty clause ensures that the author is not responsible for any damages incurred from failure in a critical system.

In addition, the license ensures that while any one can make their own changes to the original software, they cannot repackage it with a non-free license and claim the software as their own.

Finally, since software licensed under the GPLv2 must be accompanied with source code, the original author is granted full access to the source of any derived works; effectively creating a software ecosystem where developers work together to accomplish common goals.

\subsection{GNU GPLv3}

In 2007, the GPLv3 was released to work around some of the shortcomings of the original license. While this release was designed to improve upon the protections of the author, it is sometimes criticized for dividing the open source and free software communities.

\subsubsection{Tivoization}
One of the biggest changes introduced with the GPLv3 is that it blocks ``Tivoization''. Tivoization is a term invented by Richard Stallman to describe appliances that contain GPL'ed software but violate the intended freedoms of the license by making modifications impossible; often by shutting down the appliance when modifications are detected \cite{gpl3}.

While ``Tivoization''is regularly presented as a means to combat piracy, it is often used to build a monopoly on a product that uses free software. This can be good for the growth of open source software in a commercial world, but goes against what the free software movement stands for.

\subsubsection{Laws Prohibiting Free Software}

Another major change introduced with GPLv3 is that laws that criminalized the development of free software under GPLv2 no longer apply. This includes legislation such as the Digital Millennium Copyright Act and the European Union Copyright Directive.

The philosophy behind this change is that you shouldn't be criminalized for writing software, no matter the intent. While this does not forbid DRM in GPLed software, section 3 of the GPLv3 license states that ``the system will not count as an effective technological ``protection'' measure'' \cite{gpl3}. This effectively guarantees that if the author distributes software that breaks the DRM, they will not be threatened laws such as the DMCA.

\subsubsection{Patent Threats}

Finally, GPLv3 aims to avoid discriminatory patent deals. In the past companies like Microsoft and Novel have tried to use their thousands of patents to claim royalties on users for patent infringement. GPLv3 avoid this by providing explicit patent protection from the program's contributors and redistributors \cite{gpl3}.

\newpage

\subsection{GNU GPL Variations}

In addition to the main GPL license, two modified version exist which address specific problems related to the the primary GPL license.

The GNU Lesser General Public license was designed to allow developers and companies to link LGPL libraries into projects not licensed under a GPL license, even if the project contains proprietary software. \cite{lgpl}. This difference allows companies to include free software in their projects without being bound by the required terms of a strong copyleft license.

The GNU Affero General Public License was designed to close a perceived application service provider loophole in the original GPL. It includes an additional provision which requires that the complete source code be made available to any network user of the AGPL licensed work \cite{agpl}.

\subsection{MPL}

The Mozilla Public License is a weak copyleft license that aims to strike a balance between the BSD and GPL licenses. By striking this balance, the MPL motives collaboration between business and the open source community. The MPL is used in various open source projects including; Thunderbird, Firefox, and Libreoffice.

Much like the LGPL, the MPL allows mixing source code of any license, even those that are proprietary, as long as the source code that is distributed under the MPL remains under the MPL. The biggest difference between the LGPL and the MPL is that the MPL applies copyleft on the context of files while the LGPL applies copyleft on the context of libraries \cite{mpl}.

\subsection{Other Copyleft Licenses}

The Common Public License is a license produced by IBM which encourages open source development in commercial settings, while still maintaining the ability to use software licensed under separate terms. Unlike the MPL, the CPL is a strong copyleft license, sharing much of the same terminology in the GPL, however it lacks compatibility with all versions of GPL due to ``choice of law'' section which restricts legal disputes to specific courts.

The Eclipse Public License is a slightly modified version of the Common Public License which drops the terms related to patent litigations. It is used in the Eclipse IDE and various other Java related projects including the JVM and Clojure.

The Common Development and Distribution License, produced by Sun Microsystems, is based on the MPL.

The Apple Public Source License was the original license under which Apple's Darwin operating system was developed. While APSL is fully copyleft and the FSF supports the license, they do no longer recommend it. This license was essential in the first stages of Apple, however today most of the software released by the company now use the Apache license.

\newpage

\section{Deciding on a License}

While licensing software is a crucial step in any kind of open source development, it can often be difficult to determine which license best suits your project. As previously outlined, the biggest divide among the open source licenses is the difference between a permissive license and a copyleft license. If you plan on linking your open source software with proprietary software, a permissive or weak copyleft license would be the best choice. However, for many  open source projects, a copyleft license is often more suitable.


\begin{figure}[h]
\begin{adjustwidth}{-1in}{-1in}
  \centering
  \includegraphics[width=1.3\textwidth]{license-adoption}
\end{adjustwidth}
  \caption[Top 20 Open Source Licenses]{Retrieved from BlackDuck \cite{license-stats}}
\end{figure}

Considering the license adoption statistics as outline in figure 1, it is quite obvious that the GPL and its variants dominate the open source license market. However, there is massive divide between GPL and the other copyleft licenses, while the permissive licenses are distributed more evenly.

As a result of this large divide, for free software the decision almost always comes down to the GPLv3 or GPLv2. For open source software, it is ideal to choose a GPL license, but if you cannot ahere to the strict requirements of a copyleft license, it usually comes down to either the MIT, Apache or ``modified'' BSD license.

\section{Conclusion}
At first sight, licensing can seem quite daunting for those who don't care much for legal matters, but its crucial aspect of open source software development. Without a software license you can't create an open source community.

\newpage

\begin{thebibliography}{20}

\bibitem{apache1} The Apache Software Foundation. Apache License, Version 1.0 \url{https://www.apache.org/licenses/LICENSE-1.0}

\bibitem{apache11} The Apache Software Foundation. Apache License, Version 1.1 \url{https://www.apache.org/licenses/LICENSE-1.1}

\bibitem{apache2} The Apache Software Foundation. Apache License, Version 2.0 \url{https://www.apache.org/licenses/LICENSE-2.0}

\bibitem{boost} Boost C++ Libraries. Boost Software License. \url{http://www.boost.org/users/license.html}

\bibitem{nolicense} ChooseALicense.com. No License. \url{http://choosealicense.com/no-license/}

\bibitem{billyg} DigiBarn Computer Museum. Bill Gates' Open Letter To Hobbyists. \url{http://www.digibarn.com/collections/newsletters/homebrew/V2_01/gatesletter.html}

\bibitem{originalbsd} Free Software Directory. License:BSD 4Clause. \url{directory.fsf.org/wiki/License:BSD_4Clause}

\bibitem{simplifiedbsd} Free Software Directory. License:FreeBSD. \url{http://directory.fsf.org/wiki?title=License:FreeBSD}

\bibitem{x11} Free Software Directory. License:X11. \url{http://directory.fsf.org/wiki/License:X11}

\bibitem{fsfhistory} GNU Operating System. About the GNU Project \url{https://www.gnu.org/gnu/thegnuproject.html}

\bibitem{netbsd} GNU Operating System. BSD License Problem. \url{https://www.gnu.org/philosophy/bsd.html}

\bibitem{gpl2} GNU Operating System. GNU General Public License, version 2. \url{https://www.gnu.org/licenses/gpl-2.0.html}.

\bibitem{licenselist} GNU Operating System. Various Licenses and Comments About Them. \url{https://www.gnu.org/licenses/license-list.html}

\bibitem{copyleft} GNU Operating System. What is Copyleft? \url{https://www.gnu.org/copyleft/}.

\bibitem{fourfreedoms} GNU Operating System. What is Free Software? \url{https://www.gnu.org/philosophy/free-sw.html}

bibitem{gpl3} GNU Operating System. A Quick Guide to GPLv3. \url{https://www.gnu.org/licenses/quick-guide-gplv3.html}.

\bibitem{openbsd} OpenBSD. OpenBSD Copyright Policy. \url{http://www.openbsd.org/policy.html}

\bibitem{academic} Open Source Initiative. Academic Free License 3.0. \url{http://opensource.org/licenses/AFL-3.0}

\bibitem{attrassur} Open Source Initiative. Attribution Assurance License. \url{http://opensource.org/licenses/AAL}

\bibitem{education} Open Source Initiative. Education Community License, Version 2.0 \url{http://opensource.org/licenses/ECL-2.0}

\bibitem{fair} Open Source Initiative. Fair License (Fair). \url{http://opensource.org/licenses/Fair}

\bibitem{permissive} Open Source Initiative. Frequently Answered Questions. \url{http://opensource.org/faq}

\bibitem{opensourcehistory} Open Source Initiative. History of the OSI. \url{http://opensource.org/history}

\bibitem{mit} Open Source Initiative. The MIT License (MIT). \url{http://opensource.org/licenses/MIT}

\bibitem{opensourcedefinition} Open Source Initiative. The Open Source Definition. \url{http://opensource.org/definition}

\bibitem{sqlite} SQLite. SQLite Copyright. \url{https://www.sqlite.org/copyright.html}

\bibitem{copyright} Teaching Copyright. Copyright Frequently Asked Questions. \url{http://www.teachingcopyright.org/handout/copyright-faq}

\bibitem{wtfpl} WTFPL – Do What the Fuck You Want to Public License. Frequently Asked Questions. \url{http://www.wtfpl.net/faq/}


\bibitem{lgpl} GNU Operating System. GNU Lesser General Public License. \url{https://www.gnu.org/licenses/lgpl.html}.

\bibitem{agpl} GNU Operating System. Why the Affero GPL. \url{https://gnu.org/licenses/why-affero-gpl.html}.

\bibitem{mpl} Mozilla. MPL 2.0 FAQ. \url{https://www.mozilla.org/MPL/2.0/FAQ.html}

\bibitem{license-stats} Black Duck. Top 20 Open Source Licenses. \url{https://www.blackducksoftware.com/resources/data/top-20-open-source-licenses}.

\end{thebibliography}

\end{document}
